\documentclass[dvipdfmx]{jsarticle}
\usepackage[papersize={13cm,9cm},margin=1mm,noheadfoot]{geometry}
\usepackage{tikz}
\usepackage{tikz-3dplot}
\usepackage[RPvoltages]{circuitikz}
\usepackage[nomessages]{fp}
\pagestyle{empty}

\begin{document}

% 定数
\def\pi{3.14}
\def\T{1} % 描画する時間. 周波数は1に決め打ちしている。1周期分を描画すれば十分なのでT=1とする
\FPeval{omg}{clip(2*\pi)} % 角周波数(周波数は1としている)

% 軸の長さ
\def\xmin{-2}
\def\xmax{2}
\def\ymin{-1.5}
\def\ymax{1.5}
\def\zmin{0}
\def\zmax{5}
\def\zOffset{2} % z軸の先端から波の終端までの余裕を持たせる

% 時間軸のポイント数
\def\Nt{40} % pdfのページ数に等しくなる
\FPeval{dt}{clip(\T/\Nt)}

% z軸のポイント数
\def\Nz{25}
\FPeval{dz}{clip(\zmax/\Nz)}

% 電界ベクトルのパラメータ
\def\Ax{1} % x成分の振幅
\def\Ay{1} % y成分の振幅
\def\deltaDeg{0} % [deg] 電界のx,y成分の位相差

% 直線偏光を仮定して直線の傾き角 theta を計算
\FPdiv{\tanVal}{\Ay}{\Ax} % tanVal = Ay / Ax
\FParctan{\thetaLinearRad}{\tanVal} % thetaLinear = atan(Ay/Ax) [rad]
\FPeval{\thetaLinearDeg}{\thetaLinearRad*180/\pi} % [deg]

% 電界のx,y成分の位相差が180度なら
% 傾き角は theta = 180 - atan(Ay/Ax)
\FPifeq{\deltaDeg}{180} \FPeval{\thetaLinearDeg}{180-\thetaLinearDeg} \else \fi

% 波長, 波数
\def\lmd{2} % zmax / lmd 個分だけ波が描画される
\FPeval{k}{clip(2*\pi/\lmd)}

% =========== パラメータ値を確認するためのページ ===========
% % 確認が取れたらコメントアウトしてください
% $\omega=\FPprint\omg$\\
% $T=\FPprint\T$\\
% $dt=\FPprint\dt$\\
% $k=\FPprint\k$\\
% $\delta=\FPprint\deltaDeg^\circ$\\
% $\theta=\FPprint\thetaLinearRad$ rad = $\FPprint\thetaLinearDeg^\circ$\\
% \newpage
% =======================================================

% カメラアングル
\tdplotsetmaincoords{70}{140}

% tikzset
\tikzset{axis/.style={black,thick,>=Stealth,->}}
\tikzset{small_axis/.style={thick,>=Stealth,->}}
\tikzset{arrow/.style={>=Stealth,->}}

% dt おきに描画していく
\foreach \tidx [evaluate=\tidx as \t using {\tidx*\dt}] in {1,2,...,\Nt}{%
    \centering
    \begin{tikzpicture}[tdplot_main_coords,scale=2]
        \tdplotsetrotatedcoords{90}{90}{90}
        \draw[axis,tdplot_rotated_coords] (\xmin,0,0) -- (\xmax,0,0) node[anchor=south west]{$x$};
        \draw[axis,tdplot_rotated_coords] (0,\ymin,0) -- (0,\ymax,0) node[anchor=south]{$y$};
        \draw[axis,tdplot_rotated_coords] (0,0,\zmin) -- (0,0,\zmax+\zOffset) node[anchor=west]{$z$};
    
        % z軸方向に電場ベクトルの先端が描く軌跡
        \draw[blue,thick,variable=\z,domain=0:\zmax,samples=200,tdplot_rotated_coords] plot 
        ({\Ax*cos((\omg*\t - \k*\z) r)},{\Ay*cos((\omg*\t - \k*\z) r + \deltaDeg)},\z);
    
        % 各zにおける電場ベクトル
        \foreach \zidx [evaluate=\zidx as \z using {\zidx*\dz}] in {0,1,...,\Nz}
        {
            \draw[arrow,blue,tdplot_rotated_coords] (0,0,\z) --++ 
            ({\Ax*cos((\omg*\t - \k*\z) r)},{\Ay*cos((\omg*\t - \k*\z) r + \deltaDeg)},0);
        }
    
        % z = zmax における電場ベクトルの先端が描く軌跡
        \filldraw[draw=black,fill=lightgray,fill opacity=0.4,very thick,variable=\tt,domain=0:\T,samples=100,tdplot_rotated_coords] plot ({\Ax*cos((\omg*\tt - \k*\zmax) r)},{\Ay*cos((\omg*\tt - \k*\zmax) r + \deltaDeg)},\zmax);

        % x,y軸に平行な補助線
        \draw[black,thin,tdplot_rotated_coords] (-1,0,\zmax) -- (1,0,\zmax);
        \draw[black,thin,tdplot_rotated_coords] (0,-1,\zmax) -- (0,1,\zmax);

        % linear polarization
        \fill[lightgray,opacity=0.4,tdplot_rotated_coords] (-1,-1,\zmax) -- (1,-1,\zmax) -- (1,1,\zmax) -- (-1,1,\zmax) -- cycle;
        \draw[black,variable=\th,domain=0:\thetaLinearDeg,samples=100,tdplot_rotated_coords] plot ({0.2*cos(\th)},{0.2*sin(\th)},\zmax);
        \node[tdplot_rotated_coords] at (0.4,0.1,\zmax) {$\theta$};

        % z = zmax における電場ベクトルのみ色を変える
        \draw[arrow,red,very thick,tdplot_rotated_coords] (0,0,\zmax) --++ ({\Ax*cos((\omg*\t - \k*\zmax) r)},{\Ay*cos((\omg*\t - \k*\zmax) r + \deltaDeg)},0);
    \end{tikzpicture}
    \newpage
}

\end{document}